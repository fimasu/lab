% set home directory
\providecommand{\homedir}{..}
% load the preamble of main.tex by subfiles
\documentclass[\homedir/main.tex]{subfiles}

\begin{document}
\chapter*{概要}\label{chap:abstract}%TODO
現代では,我々はしばしば色やエフェクトなどによって装飾された文字を目にする.
このように装飾された文字は「デザイン文字」と呼ばれる.
デザイン文字は装飾のないシンプルな文字と比べ,
目立ちやすく具体的なイメージを喚起しやすいという利点がある.
そのため,作品のタイトル・企業名・商品名などのロゴや,チラシ・ポスターなどの広告のように
特に目立たせたい文字をデザイン文字にすることが多い.

一方,デザイン文字の作成には高度な技術や多大な時間を要するという欠点がある.
そのため,デザイン文字の簡単な作成手法が求められており,
Style Transferを用いて自動作成する手法が
既にいくつか研究・開発されている.

また,「デザイン文字を用いてまで印象を強めたい文字」は,
固有名詞や何かのメッセージのような意味のある文字列の一部であることが多い.
つまり,デザイン文字が用いられる際には単一の文字としてではなく,
複数文字からなる文字列として用いられることが多いと予想できる.

ゆえに,「デザイン文字」の作成手法よりも「デザイン文字列」の作成手法の方が有用だと考え,
「デザイン文字列を自動で作成する手法の開発」を本研究の目的とした.

なお,デザイン文字列には単一の文字からなるデザイン文字では考慮する必要のなかった
デザインの要素がいくつか存在する.
そういった要素の例として,文字の間隔や配置,文字列全体の形状の歪みが挙げられる.
これらの要素はより目を引くデザイン文字列を作成するために用いられる.

こうしたデザイン文字列特有のデザインの要素のうち,
本研究では「文字列全体の形状の歪み」に射影変換や区分線形回帰を用いて対応することを試みた.

\end{document}
