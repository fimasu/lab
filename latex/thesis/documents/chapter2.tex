% set home directory
\providecommand{\homedir}{..} 
% load the preamble of main.tex by subfiles
\documentclass[\homedir/main.tex]{subfiles}
% ##############################################################################
\begin{document}
% set chapter numbering to work correctly even when separate compilation using subfile
\setcounter{chapter}{1}
% ##############################################################################
\chapter{関連研究}\label{chap:related_works}
本研究ではStyle Transferを用いてデザイン文字列を生成している.(\cref{chap:methods}で詳述する.)
そこで本章では,まず\cref{sec:style_transfer}でStyle Transferについて述べる.
次に,本研究の関連研究として
\cref{sec:text_effects_transfer}でStyle Transferをデザイン文字に応用した研究を紹介し,
\cref{sec:logo_style_transfer}でStyle Transferをロゴに応用した研究を紹介する.
本章の最後に,\cref{sec:positioning}でこれらの先行研究を踏まえた本研究の位置づけについて述べる.

% ##############################################################################
\section{Style Transfer}\label{sec:style_transfer}
\subsection{Style Transferとは}
Style Transferとは「対象の構造を決定づけるデータ」と
「対象の見た目や雰囲気を決定づけるデータ」の二種類を上手く融合させ,
新たなデータを生成する技術である.
この二種類のStyle Transferへの入力データのうち,「対象の構造を決定づけるデータ」は「コンテンツ」と呼ばれ,
「対象の見た目や雰囲気を決定づけるデータ」は「スタイル」と呼ばれる.

例えば,Style Transferの主要なタスクの一つである画風変換では,
「画像中の物体の輪郭や配置」が「コンテンツ」に相当し,
「画像の色使いやタッチ」が「スタイル」に相当する.

\subsection{畳み込みニューラルネットワークを用いた画風変換手法}
Gatysら\cite{Gatys_2016_CVPR}は
畳み込みニューラルネットワークを用いて画風変換を行う手法を提案した.
当時,Image Analogies\cite{Hertzmann_2001}のように
ニューラルネットワークを用いない画風変換手法は既に存在していたが,
画風変換にニューラルネットワークを用いるのは先駆的な試みであった.
この手法の発表以後もニューラルネットワークを用いない手法\cite{Elad_2017}が
研究されなくなったというわけではないものの,
ニューラルネットワークを用いる手法\cite{Johnson_2016, Li_2017}が主流となった.

% ##############################################################################
\section{Style Transferとデザイン文字}\label{sec:text_effects_transfer}

Yangら\cite{Yang_2017_CVPR}は,
デザイン文字におけるStyle Transferという問題を提起し,
これを解決する手法としてパッチマッチングを応用した手法を提案した.

この研究はデザイン文字のエフェクトと
Style Transferをデザイン文字に応用した研究の中では重要な研究の一つである.
なぜなら,\cite{Yang_2017_CVPR}の主要な結果の一つである,文字の
文字の骨格部分,すなわち文字を構成する線の中心線の集まりである「スケルトン」
との距離とエフェクトに対応関係があることを明らかにし後の研究に影響を与えたためである.


Wangら\cite{typography2019}は,


Yangら\cite{Yang2019Controllable}は,

Yangら\cite{Yang2019TETGAN}は,

% ##############################################################################
\section{Style Transferとロゴ}\label{sec:logo_style_transfer}%TODO
\subsection{企業ロゴにヘヴィメタルバンドのロゴのスタイルを転写する研究}
Ter-Sarkisov.\cite{icpram20}は,

\subsection{Style Transferを用いてロゴを自動で作成する手法の研究}%TODO
アタルサイハン\cite{Atarsaikhan}は,

% ##############################################################################
\section{本研究の位置づけ}\label{sec:positioning}
本章でここまで述べてきたように,
「デザイン文字のStyle Transfer」に関する研究は多く存在する.
しかし,\cref{sec:background}で述べたようなデザイン文字列に特有のスタイルは
先行研究ではあまり注目されてこなかった.
そこで,本研究ではデザイン文字列に特有の要素のうち形状の歪みに着目した.
この形状の歪みは本章で述べてきたような手法では扱いづらいが,
射影変換や回帰分析を用いれば上手く扱うことができる.
そのため,先行研究によって研究・開発されてきたデザイン文字のStyle Transferの手法に,
射影変換や回帰分析を用いて形状の歪みを扱う手法を組み合わせることで,
本研究では形状の歪みのあるデザイン文字列を自動作成する手法を開発することを目指す.
% ##############################################################################
\printBibForSubfiles
\end{document}