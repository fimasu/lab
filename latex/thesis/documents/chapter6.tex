% set home directory
\providecommand{\homedir}{..} 
% load the preamble of main.tex by subfiles
\documentclass[\homedir/main.tex]{subfiles}
% ##############################################################################
\begin{document}
% set chapter numbering to work correctly even when separate compilation using subfile
\setcounter{chapter}{5}
\chapter{結論}\label{chap:summary}

\section{まとめ}
本研究では形状の歪みのあるデザイン文字列を自動で作成する手法として
射影変換とStyle Transferを組み合わせて自動作成する手法を開発した.

\section{今後の課題}
歪みのスタイルを指定する文字列画像に漢字などの一部の文字種を用いた場合,
歪みの転写が失敗することへの対策が必要である.
また,Style Transferを用いてエフェクトのスタイルを転写するする過程において
画像のリサイズをおこなうことに起因する画質劣化への対策が必要である.
さらに,本研究では扱わなかったデザイン文字列に特有なデザインの要素である,
文字の配置などを上手く扱う手法を開発することも今後の課題である.

\printBibForSubfiles
\end{document}