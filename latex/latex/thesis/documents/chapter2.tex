% set home directory
\providecommand{\homedir}{..} 
% load the preamble of main.tex by subfiles
\documentclass[\homedir/main.tex]{subfiles}
% ##############################################################################
\begin{document}
% set chapter numbering to work correctly even when separate compilation using subfile
\setcounter{chapter}{1}
\chapter{関連研究}\label{chap:related_works}
本研究ではStyle Transferを用いてデザイン文字列を生成している.(\cref{chap:methods}で詳述する.)
そこで,本章では本研究の関連研究として,
\cref{sec:neural_style_transfer}でNeural Networkを用いたStyle Transferに関する研究を紹介し,
\cref{sec:text_effects_transfer}でStyle Transferをデザイン文字に応用した研究を紹介し,
\cref{sec:logo_style_transfer}でStyle Transferをロゴに応用した研究を紹介する.
本章の最後に,\cref{sec:positioning}でこれらの先行研究を踏まえた本研究の位置づけについて述べる.

\section{Neural Networkを用いたStyle Transfer}\label{sec:neural_style_transfer}
そもそも,Style Transferとは「対象の構造を決定づけるデータ」と
「対象の見た目や雰囲気を決定づけるデータ」の二種類を上手く融合させ,
新たなデータを生成する技術である.
この二種類のStyle Transferへの入力データのうち,「対象の構造を決定づけるデータ」は「コンテンツ」と呼ばれ,
「対象の見た目や雰囲気を決定づけるデータ」は「スタイル」と呼ばれている.
Style Transferの主要なタスクの一つである画風変換を例にとると,
「画像中の物体の輪郭や配置」が「コンテンツ」にあたり,
「画像の色使いやタッチ」が「スタイル」にあたる.
つまり,Style Transferによる画風変換は「ある画像中の物体の輪郭や配置を可能な限り保存しつつ,
色使いやタッチを別の画像のものに似せるタスク」だといえる.

現在,このStyle Transferによる画風変換の分野では,
Neural Networkを用いたアプローチが盛んに研究されている.
そのきっかけとなったのが\cite{Gatys_2016_CVPR}である.


\section{文字の効果や装飾のStyle Transfer}\label{sec:text_effects_transfer}
Neural Networkを用いてはいないが重要
\cite{Yang_2017_CVPR}

文字の効果や装飾をStyle Transferによって転写する研究がある
\cite{Yang2019Controllable}
\cite{typography2019}
\cite{Yang2019TETGAN}

\section{Style Transferとロゴ}\label{sec:logo_style_transfer}
Style Transferを用いてロゴ作成を行う研究がある

\cite{icpram20}

\cite{oai:irdb.nii.ac.jp:01211:0005350653}


\section{本研究の位置づけ}\label{sec:positioning}
以上に述べたような先行研究では,
文字列を含んだデザインの作成は技術や時間を要するが,
スタイル転写を応用することでこれを支援することを目的とする.
また、文字列を含んだデザインには歪みがあることがある.
こうした歪みは単にスタイルの転写を行うだけでは転写できない.
そこで、歪みのあるデザインからの歪みの推定・転写を可能にし、
単なるスタイル転写では扱えないものに対応することを目指す.


\printBibForSubfiles
\end{document}